%!TEX TS-program = xelatex
\documentclass[]{class}
% \addbibresource{bibliography.bib}


% Define the column separation size in tabular
\setlength{\tabcolsep}{6pt}


\begin{document}

% \thispagestyle{plain}

Université de Sherbrooke\hfill
Master Studies
\par
\rule{\textwidth}{1pt}
\par\vspace{3pt}
\begin{center}
  \huge\textbf{\color{black}Trend quantifier system}
\end{center}
\par\vspace{3pt}
\rule{\textwidth}{1pt}

\par\vspace{15pt}

\begin{large}
  \begin{center}
  \begin{tabular}{c c}
    \textsc{Date:}& \textsc{\today}\\
    \textsc{Due :}& \textsc{January 3, 2016}\vspace{3ex}
  \end{tabular}
  \end{center}
\end{large}
% \header{Romain}{Cotte}{Ingénieur informatique}

% \par\vspace{20pt}

\section{Problem definition}
{\large\textbf{Goal:}} Given a list of words, determine and quantify their trend. Find the associated trends.\\
\par
{\large\textbf{Required tasks:}} You are in charge of developing
\begin{itemize}
  \item an engine which scraps the twitter website or consume the twitter api in order to store the number of word occurrences
  \item a website to display results in graphics
\end{itemize}

\par\vspace{5pt}
{\large\textbf{Bonus tasks:}}
\begin{itemize}
  \item suggest tweets for user
\end{itemize}


% Interface:
% Elle devra comprendre une barre de recherche :


 . Study the twitter api. Define possibilities.
 . Take several tweets and save it in the database (it would be a base for the future development)
 . Develop a twitter api consumer
 . Analyse tweets,
 . Develop website


\newpage

\begin{thebibliography}{1}

  \bibitem{google-trend-example} Google trend example \url{https://www.google.com/trends/explore#q=Sherbrooke}

  \bibitem{twitter-api-doc} Twitter API documentation \url{https://dev.twitter.com/overview/documentation}

\end{thebibliography}

\end{document}


